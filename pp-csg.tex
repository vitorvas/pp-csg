\documentclass{elsarticle}

\author{Vitor Vasconcelos, Andr\'e Campagnole, Daniel Campolina and 
Graiciany Barros}
\title{A visualizer for Serpent Nuclear Code geometries}

\begin{document}

\maketitle

\section{Why bother}

I can cite my professor Claudio Esperanca\cite{Esperanca1990}.
It is also worth citing the Los Alamos White Paper\cite{Spencer2017}.
The paper of Boyd\cite{Boyd2015} presents a nice open software worth citing.
In this paper (\textbf{which paper? The one of Boyd or this one I'm writing?}), a new
library for CS (or CSG) graphic elements is described. The proposed work does not investigate better
implementations or algorithms for CSG, but clear explanations of the algorithms used in general Monte Carlo
codes are given. These algorithms and data structures can be used in the present work and allied to
VTK.

The decision of which language to use was made: C++. I chose it over Python in order to make use of examples, try to polish
my C++ and, I might be wrong here, but I think it is more straightforward to use a C++ test tool. In the present case, I'm starting
to play with Gtest (Google Tests Framework)[cite].

Another main issue before even starting to code: how \textbf{Serpent2} implements the geometry/elements functions. This
can be the best reference for implementing the visualization software. \textbf{Answer:} well, this is no big deal. Since the
implementation is not a CSG library, no need to provide the same operations of a truly CSG implementation. The idea is to
provide the operations needed to generated 3D models of Serpent input files, no matter it is a real CSG operation or not.

\section{Introduction (Justification)}

There is no question about the usefulness of Monte Carlo codes to the various
fields of radiation applications, namely, nuclear physics, nuclear reactors,
radio-protection, doses assessment, among many others \cite[applications].

A lot of time and energy were spent implementing algorithms intended to improve
various aspects of the physics simulated by these codes. The most popular of
them is MCNP\textregistered \cite{mcnp}, followed by Fluka\cite{fluka} and, more
recently, Serpent\cite{Leppanen2015}. The latter, aimed mostly to criticality and neutron analysis.

The maturity, robustness and applicability of these codes cannot be questioned.
However, since resources for software development are also a limited asset,
usually these are applied to add new features concerning mostly the physical aspects of the software and the usability is usually neglected.

The urgent need for a better form of software interaction related to MCNP
is well presented by Spencer\cite{Spencer2017}, taking in account a set of gains that would emerge simply from the improvement of the way on how geometry definition is done for MCNP.

Despite being a much younger software than MCNP, Serpent has at least one limitation common to MCNP: the description of the problem's geometry. This affirmation is by no means a criticism to its input standard, since it has a comprehensive form of geometry
definition. Nevertheless, this does not mean neither that it cannot be improved.

With that in mind, this paper presents a rather simple - and unexpectedly expansive -
graphics user interface aimed to the visualization of Serpent's geometry input
in 3D. This version generates VTK output files to be read and manipulated by
the increasingly present scientific visualization tool Paraview\cite{paraview}.

\section{Objective}


\section{CSG on nuclear codes}

A fundamental reference for the present work is the former development of a CSG
(called Combinatorial Geometry - CG - in their work) library for geometry construction
of nuclear models. The \textit{OpenCG} library \cite{Boyd2015} is used by both
\textit{OpenMC}\cite{Romano2015} and \textit{OpenMOC}\cite{Boyd2014} software, developed by
MIT's Computational Reactor Physics Group.

\textit{OpenCG} has a wider and different set of applications than the
software presented in this work. Its main objective - as attested by the authors -
is to provide, among other applications, verification of results between transport
codes. Put simple: \textit{OpenCG} is a way of generating complex
Combinatorial geometries once and have different input files generated for the main
Monte Carlo nuclear codes thus allowing for verification.

The objective of the present work is far more simple. Instead of generating the inputs for
different codes, our program goes on the opposite direction, reading different
inputs\footnote{At present only inputs for Serpent and MCNP.} and generating
a 3D graphical model representing this geometry. The workflow of our code is presented
in figure \ref{fig:workflow}.

\section{Why do not use existent CSG libraries?}

There are few open or free CSG libraries available (cite them). However, none of them were considered
the best choice for this project. The reason is mainly simplicity: this project does not need to implement
all the basics of CSG operations, but only those related to the geometries used by the nuclear codes
source of data. Another reason is a bias in favor of VTK (cite), since it is the natural library of visualization
of data in Paraview (cite), it has a solid community of users and is under constant development.

\section{Geometry in Serpent}

As many Monte Carlo particle simulation codes, Serpent makes use of CSG to
describe the geometry of the elements being simulated. Serpent calls it: universe-based geometry modeling.
%Besides the conventional square and hexagonal lattices used in nuclear reactor %simulations, Serpent also
%offers a special geometry type for the simulation of CANDU type reactors and also randomly-dispersed
%particle fuels.

%Serpent is the first nuclear reactors Monte Carlo code to natively support CAD and %unstructured mesh based geometries\cite{serpent-openfoam-mesh}, which are not 
%(yet) in the scope of the implementation presented in this paper.

%The building block 
%of geometry is the cell, which is a region of space delimited by boundary 
%surfaces. Cells are filled with 

\subsection{How \textit{Serpent} defines geometry?}

Serpent has a simple yet powerful way of defining geometries. Serpent has 
surfaces, cells and universes. In Serpent, surfaces can be elementary or 
derived. Derived surfaces are those formed by two or more elementary surfaces. 
Serpent offers forty-two (42) different derived surface types \ref{tab:surfaces} for the convenience of the user. Surfaces are numbered to allow identification.

Cells, which can be two or three dimensional, are a delimited part of the space 
formed by surfaces. Cells have names (or numbers) to identify themselves and 
each cell belong to a universe. Cells can be said \texttt{void}) if 
empty, \texttt{outside} if describing a region of space that is not part 
of the model's geometry of filled by another universe. In the latter case the 
keyword used is \texttt{fill}.

But a fundamental concept remains open: what is a universe? A universe is a "filling pattern". In other words, a universe is a domain covering the space where it is defined. Despite the awkward definition, a universe allow the geometry to 
be divided into different separated levels. As a constraint, a universe must 
cover all space in where it is defined. It is, by default, centered at the 
origin and can be rotated and translated at the users will in order to allow 
the construction of complex geometries.

With the concept of universes in mind, is easy to introduce the concept of 
lattices. Lattices are a special type of universes, which are filled with 
regular structures of other universes. Serpent has eight type of lattices. 
Figure \ref{fig:lattice-example} shows a regular lattice [...].


%Serpent has 42 (forty-two) types of surfaces. Their names and the associated
%list of parameters which are used to propertly define them are stored
%in separated data strctures, one holding surfaces names and other keeping
%the parameter list for each surface. These variables are defined in the
%header file \texttt{surface\_types.h}\footnote{It is worth mention
%  that parameters of surfaces are stored in an array of arrays of size
%  two, holding tuples representing the minimum and maximum number of parameters
%  accepted for each surface.}.

% Maybe make a table showing the possible surfaces, theis parameters
% and a short description (image)

\subsection{Features and limitations}

\section{Development (libraries)}

The chosen toolkit used to develop a visualization software for Serpent input 
geometries is the Visualization Toolkit (VTK)\cite{vtk}. It is an open-source, 
freely available software system for, among other applications, visualization of 
scientific data. In the present case, this data is a set of geometrical 
primitives used to model the geometry of a real problem, specifically, a 
nuclear reactor core.

%\begin{itemize}
%    \item Something I'm wondering: once I can intersect and build elements like in Serpent/MCNP, should I make
%      them polydata like unstructured grids?
%    \item I maybe should try a first version of code in Python and then code the well-tought version in C++. This would make double the work, but I would practice both languages the way I want.
%    \item Attention to messages marked as \textbf{important} in VTK users list, they might be useful to the future problems I'll face when implementing.
%\end{itemize}

\section{Methodology}

After giving the first try, i.e, having a working example using VTK with some data structures working, clipping functions and writing
\textbf{vtk/vtm} files, it is possible to write some thoughts on how it started and might continue.

The very basic region in a Serpent simulation is the \textit{universe}. This is simulation space and, in Serpent, can be a cube or a 3D hexagon.
The first implementation builds a cubic \textit{vtkUnstructuredGrid} from a dimensions in x, y, and z. At this point, it is not even a function,
but a loop. This loop builds a list of points stored as vtkPoints and the cells, which are hexahedra. Each cell made of 8 points. For now, the
loop works for any number of cells in any directions. The next step is to make it a function or a class (need to think about it) and any methods
it will provide. If it will only provide points/cells or it will hold the unstructured grid and give access to it. Need some thought.

At this point, it is necessary to starting to think about tests. The general idea of the algorithm too. The tests are not written, but their
CMakeLists.txt file is written and they're compiled with the project. The Gtest is being used, but I must read about Ctests which are native
to CMake.

So, to organize the methodology:
\begin{enumerate}

\item Make the base geometry generator a function/class depending on how the program will work;
\item Write tests for this class. What to test and how to test.
\item Establish coding standards (syntax) to be used.
\item Read FOSS journal to check what is needed to publish with them.
\item \textbf{Rewrite the code} the cleanest way possible and make it git controlled in \textbf{gitlab}\footnote{Use gitlab to be able to keep the code hidden until release (GPL, probably) and try to use gitlab test features to automate tests.}
\item Implement the parser for Serpent files.
\end{enumerate}

This visualizer has potential to be used also for MCNP so, would be a better bet to have a MCNP parser too before releasing the code. This will
make better chances to be cited.

\section{Examples of use}

\section{Conclusions and Future work}

\bibliographystyle{plain}
\bibliography{pp-csg}

\end{document}

