\documentclass{elsarticle}

\author{Vitor Vasconcelos}
\title{A visualizer for Serpent Nuclear Code geometries}

\begin{document}

\maketitle

\section{Why bother}

I can cite my professor Claudio Esperanca\cite{Esperanca1990}.
It is also worth citing the Los Alamos White Paper\cite{Spencer2017}.

\section{Introduction (Justification)}

There is no question about the usefulness of Monte Carlo codes to the various
fields of radiation applications, namely, nuclear physics, nuclear reactors,
radioprotection, dosis assesment, among many others \cite[applications].

A lot of time and energy were spent in implementing algorithms intend to improve
many aspects of the physics simulated by these codes, being the most used of
them MCNP\textregistered \cite{mcnp}, followed by Fluka\cite{fluka} and others, like the more
recent Serpent\cite{serpent} aimed mostly to criticality and neutron analysis.

The maturity, robustness and aplicability of these codes cannot questioned.
However, since resources for sofware development are also a limited asset,
usually these are applied to the physical aspects of the software and
the usability is usually a neglected aspect of them.

The urgent need for a better form of software interaction related to MCNP
is well presented by Spencer\cite{Spencer2017}, taking in account a set of gains that
would be possible by improving the geometry definition for MCNP.

Despite being a much younger software than MCNP, Serpent has at least one limitation
common to MCNP: the description of the problem's geometry. This affirmation is by no
means a critic to its input standard, since it has a comprehensive form of geometry
definition. Nervetheless, this does not mean neither that it cannot be improved.

With that in mind, this paper presents a rather simple - and expectedly expansive -
graphics user interface aimed to the visualization of Serpent's geometry input
in 3D. This version generates VTK output files to be read and manipulated by
the increasely present scientific visualization tool Paraview\cite{paraview}.

\section{Objective}


\section{Describe Serpent's geometry (CSG)}

As many Monte Carlo particle simulation codes, Serpent makes use of Computational Solid Geometry (CSG) to
describe the geometry of the elements being simulated. Serpent calls it universe-based geometry modelling.
Besides the conventional square and hexagonal lattices used in nuclear reactor simulations, Serpent also
offers a special geometry type for the simulation of CANDU type reactors and also randomly-dispersed
particle fuels.

Serpent is the first nuclear reactors Monte Carlo code to natively support CAD and unstructured
mesh based geometries\cite{serpent-openfoam-mesh}, which are not (yet) in the scope of the implementation presented in this paper.

\subsection{How \textit{Serpent} define geometry?}


\section{Features and limitations}

\section{Development (libraries)}

\section{Examples of use}

\section{Conclusions and Future work}

\bibliographystyle{plain}
\bibliography{pp-csg}

\end{document}

